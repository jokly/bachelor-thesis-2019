\documentclass{fefu}

\begin{document}
  \section*{Заголовок}
  Защищается студент группы Б8403а на тему "Рекомендательная система для VLru Еда".
  Руководитель: старший преподаватель кафедры информатики, математического и компьютерного
  моделирования Кленин Александр Сергеевич.

  \section{Проект ”VLru Еда”}
  VLru Еда – это удобный сервис заказа блюд из кафе и ресторанов Владивостока.
  Проект был запущен в апреле 2018 года. Сейчас он насчитывает более ста (100) компаний
  и около восьми тысяч (8000) блюд. Также, проект существует в Хабаровске под
  названием "DVHAB Еда".

  \section{Цель работы}
  Целью данной работы является разработка и внедрение на сайт VLru Еда
  рекомендательной системы.

  \section{Структура данных}
  Для построения рекомендаций была получена история заказов из базы данных,
  структура которой представлена на изображении. Таким образом, можно узнать
  купленные пользователем блюда, а также некоторую информацию о них (например:
  тег, компанию или раздел в меню).

  \section{Коллаборативная фильтрация}
  Одним из основных подходов в рекомендательных системах является коллаборативная фильтрация.
  Этот метод заключается в использовании известных предпочтений группы
  пользователей для прогнозирования неизвестных предпочтений другого пользователя.
  С помощью этой модели необходимо вычислить список, состоящий из $N$ блюд, наиболее
  релевантных каждому пользователю.

  \section{Item-based коллаборативная фильтрация}
  В данной работе была применена модель item-based коллаборативной фильтрации.
  Причины:
  \begin{itemize}
    \item При количестве пользователей большем количества оценок требуется гораздо
    меньше памяти и времени для вычисления весов подобия
    \item Так как список блюд редко меняется, относительно списка пользователей,
    можно часто переобучать модель, оставаясь при этом способными
    рекомендовать товары новым пользователям
    \item Преимущество этой модели состоит в том, что с помощью списка соседей у
    товара и их весов схожести можно объяснить пользователю рекомендации
  \end{itemize}

  На входе имеется таблица, в каждой ячейки которой находится оценка пользователя товару.
  Рассмотрим функции схожести, которые отражают на сколько один товар похож на другой.

  Для вычисления схожести между двумя множествами используется Коэффициент Жаккара.
  Он применяется в случае, когда у товара есть неявный отклик. Например купил
  пользователь товар или нет.

  В случае, когда необходимо рассматривать оценки (например от 1 до 5) используется
  Косинусное сходство. Проблема заключается в том, что данная функция не учитывает
  различие оценок. Например, если двум пользователям не понравился товар, то один
  из них может поставить оценку 3, а другой 2.

  В таком случае применяется другая метрика схожести - Коэффициент корреляции
  Пирсона, в которой данный эффект устранен.

  \section{Вычисление предсказаний}
  Для каждой метрики схожести используется своя функция для предсказания рекомендации.

  В случае с Коэффициентом Жаккара предсказывается занчение в промежутке от 0 до 1,
  которое отражает на сколько данный товар может быть интересен пользователю

  В случае с Косинусным сходством и Коэффициентом корреляции Пирсона предсказывается
  оценка, поставленная пользователем.

  \section{Выбор целевой функции}
  В данной работе было рассмотрено три подхода в построении рекомендаций.

  В первом подходе рассматривается факт покупки, то есть рассматривается множество
  товаров, которые купил пользователь.

  Во втором подходе в качестве оценки берется количество заказанного товара.

  В третьем подходе в качестве оценки берется значение функции сигмоиды, где аргументом
  $x$ является количество заказанного товара.

  \section{Метрики качества}
  Для оценки качества моделей были рассмотрены две метрики.

  Precision - это доля рекомендаций, понравившихся пользователю.

  Recall - доля интересных пользователю товаров, которая показана.

  \section{Сравнение моделей}
  Для измерения точности предсказаний датасет делился в хронологическом порядке,
  то есть в тестовой выброке были заказы, сделанные позднее, чем в обучающей.

  \section{Классификация}
  Также был рассмотрен другой подход рекомендации. Можно пытаться предскзаать
  следующий товар, который пользователь добавит к себе в корзину на основе уже добавленных.
  Для этого была построена таблица, в которой значением ячеек может быть True или False, что означает
  находится товар в корзине или нет. Значение, которое мы хотим предсказать находится
  в столбце Y и является уникальным идентификатором товара.

  \section{Модели классификации}
  Для этой задачи были рассмотрены следующие модели:
  \begin{enumerate}
    \item Наивный байесовский классификатор
    \item Random Forest
    \item Стохастический градиентный спуск
    \item Градиентный бустинг
  \end{enumerate}

  \section{Разработанное решение}
  Таким образом разработано решение, в котором:
  \begin{enumerate}
    \item Выбрана модель с фактом о покупке
    \item Выбрана мера Жаккара
    \item Имеется возможность рекомендовать компании или блюда
    \item А также, рекомендовать, фильтруя по тегу или компании
  \end{enumerate}
  Одной из проблем коллаборативной фильтрации является проблема "холодного старта".
  Она заключается в том, что у нас нет данных о новых пользователях и соответственно
  мы не можем определить их вкусы. Поэтому для них рекомендуются самые популярные блюда.

  \section{Процесс развертывания}
  На данный момент в отделе активно внедряются подходы непрерывного интегрирования,
  для этого был разработан следующий процесс развертывания.

  \section{Пользовательский интерфейс}
  В ходе обсуждений было принято решение представлять рекомендации в виде списка
  компаний, для каждой из которых представлен список блюд.

  \section{Реализация}
  В ходе работы было написано и удалено следующее количество строк кода.

  \section{Заключение}
  Таким образом было:
  \begin{itemize}
    \item Проведено сравнение моделей рекомендательных систем
    \item Реализован микро-сервис рекомендаций
    \item А также, произведена интеграция на сайт "VLru Еда" и получен акт о внедрении
  \end{itemize}

  \textbf{Спасибо за внимание!}

\end{document}
