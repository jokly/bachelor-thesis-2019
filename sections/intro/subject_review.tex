В настоящее время большую популярность набирают онлайн продажи различных товаров
и услуг. Одним из таких направлений является доставка еды из различных кафе и
ресторанов. Пользователю предлагается выбрать понравившиеся блюда, оформить заказ
и оплатить его через интернет, используя только свой компьютер или телефон.

В августе 2018 года компанией «Примнет» во Владивостоке был запущен проект "VLru Еда",
а также через некоторое время и в Хабаровске "DVHAB Еда". На данный момент на сервисах
размещено более 150 компаний и около 13000 блюд. Каждый день совершается большое
количество заказов, но пользователям очень сложно сделать выбор из такого объема товаров.
Перед компанией стоит задача помочь клиенту в этом не легком вопросе. Для этого
на всем этапе от первого посещения сайта и до оформления заказа собирается информация о
пользователе и действиях, которые он совершил. С помощью этих данных решается
большой спектр задач: какие интерфейсные изменения необходимо внести, какого
функционала не хватает для комфортного использования сервиса. Но все эти изменения
не помогут в той или иной степени помочь клиенту с "правильным" выбором.

Таким образом задачей данной работы является разработка системы, которая могла бы
выявить интересы пользователя и предложить ему наиболее релевантные товары.
