Преддипломная практика ­является заключительным этапом подготовки к написанию
выпускной квалификационной работы и к работе по специальности. Студент может на
практике познакомиться со всеми аспектами деятельности программиста.

Практика предназначена для обобщения, актуа­лизации и систематизации полученных
за время обучения теоретических знаний.

Сроки прохождения практики: с 29 апреля 2019 по 22 июня 2019 (общая
длительность – 8 недель).
За время практики требуется выполнить следующие задачи:
\begin{enumerate}
  \item Изучение различных подходов в рекомендательных системах
  \item Выгрузка и предобработка данных
  \item Реализация нескольких моделей рекомендательных систематизации
  \item Сравнение моделей
  \item Реализация микро-сервиса рекомендательной системы
  \item Тестирование
  \item Написание отчета и документации
\end{enumerate}
Для их достижения был составлен план работ:
\begin{enumerate}
  \item Сбор данных и их анализ (1 неделя)
  \item Анализ моделей коллаборативной фильтрации (2 недели)
  \item Анализ моделей классификации (2 недели)
  \item Расчет качества моделей и выбор наилучшей из них (1 неделя)
  \item Реализация микро-сервиса рекомендательной системы (2 недели)
  \item Тестирование всей системы (1 неделя)
\end{enumerate}
