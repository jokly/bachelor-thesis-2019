В качестве языка программирования для написания рекомендательной системы был
выбран Python 3.6, так как с его помощью можно быстро разрабатывать программы и
он имеет большое количество библиотек для реализации поставленной задачи. Модели
рекомендательных систем были взяты из библиотеки Turicreate. Она имеет набор всех
необходимых моделей, а также реализована на языке программирования C++, что обеспечивает
быстрое обучение и предсказание рекомендаций. В качестве фреймворка для веб-сервера
был выбран Flask, он имеет все необходимы элементы для реализации API, а также
имеет меньшее время ответа относительно своих конкурентов.

На данный момент в отделе VLru активно внедряется Kubernetes, поэтому было принято решение
использовать данную технологию. Она обеспечивает отказоустойчивость системы, а также
позволяет быстро менять конфигурацию сервера (например менять допустимый объем памяти).
