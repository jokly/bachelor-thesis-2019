Тестирование проводилось на данных, собранных с апреля 2018 года. Датасет делился
на обучающую и тестовую выборку в соотношении 80\% к 20\% соответственно.

\subsubsection{Коллаборативная фильтрация}

При тестировании было использовано два подхода в делении датасета на обучающую
и тестовую выборки: случайное и деление в хронологическом порядке (в тестовой
выборке находятся заказы, совершенные позже чем в обучающей).

Также, было испытано 4 способа рекомендаций:
\begin{enumerate}
  \item Предсказывать наличие товара в заказе в промежутке от 0 до 1, где 1 - товар куплен
  \item Предсказывать количество заказанного товара
  \item Предсказывать значение сигмоиды, где аргументом $x$ является кол-во заказов
  \begin{equation}
    y = \frac{1}{1 + e^x}
  \end{equation}
  \item Строить рекомендации на основе факта покупки
\end{enumerate}
Для вычисления сходства товаров будем использовать функции, описанные в разделе \ref{mesure}.

Для оценки качества моделей будем использовать метрики, приведенные в разделе \ref{metrics}.
При расчете рекомендаций возвращается список из 10 рекомендованных товаров.


\textbf{Деление датасета случайным образом}

\begin{table}[H]
  \centering
  \begin{tabular} { | c | c | c | }
    \hline
    Тип & CS & PS \\
    \hline
    Колонка [0, 1] & 0.951 & 1.0 \\
    \hline
    Кол-во товара & 1.122 & 0.409 \\
    \hline
    Сигмоида & 0.709 & 0.089 \\
    \hline
  \end{tabular}
  \caption{RMSE}
\end{table}

\begin{table}[H]
  \centering
  \begin{tabular} { | c | c | c | c | }
    \hline
    Тип & JS & CS & PS \\
    \hline
    Колонка [0, 1] & - & 0.0435 & 0.0 \\
    \hline
    Кол-во товара & - & 0.0415 & 0.0 \\
    \hline
    Сигмоида & - & 0.0431 & 0.0 \\
    \hline
    Факт покупки & 0.0477 & 0.0432 & - \\
    \hline
  \end{tabular}
  \caption{Precision}
\end{table}

\begin{table}[H]
  \centering
  \begin{tabular} { | c | c | c | c | }
    \hline
    Тип & JS & CS & PS \\
    \hline
    Колонка [0, 1] & - & 0.242 & 0.0 \\
    \hline
    Кол-во товара & - & 0.229 & 0.0 \\
    \hline
    Сигмоида & - & 0.238 & 0.0 \\
    \hline
    Факт покупки & 0.27 & 0.24 & - \\
    \hline
  \end{tabular}
  \caption{Recall}
\end{table}

\textbf{Деление датасета в хронологическом порядке}

\begin{table}[H]
  \centering
  \begin{tabular} { | c | c | c | }
    \hline
    Тип & CS & PS \\
    \hline
    Колонка [0, 1] & 0.963 & 1.0 \\
    \hline
    Кол-во товара & 1.193 & 0.521 \\
    \hline
    Сигмоида & 0.719 & 0.094 \\
    \hline
  \end{tabular}
  \caption{RMSE}
\end{table}

\begin{table}[H]
  \centering
  \begin{tabular} { | c | c | c | c | }
    \hline
    Тип & JS & CS & PS \\
    \hline
    Колонка [0, 1] & - & 0.0312 & $8.683 * 10^{-5}$ \\
    \hline
    Кол-во товара & - & 0.0292 & $7.235 * 10^{-5}$ \\
    \hline
    Сигмоида & - & 0.0313 & $2.17 * 10^{-5}$ \\
    \hline
    Факт покупки & 0.0339 & 0.0312 & - \\
    \hline
  \end{tabular}
  \caption{Precision}
\end{table}

\begin{table}[H]
  \centering
  \begin{tabular} { | c | c | c | c | }
    \hline
    Тип & JS & CS & PS \\
    \hline
    Колонка [0, 1] & - & 0.24 & 0.0004 \\
    \hline
    Кол-во товара & - & 0.224 & 0.0003\\
    \hline
    Сигмоида & - & 0.24 & $5.556 * 10^{-5}$ \\
    \hline
    Факт покупки & 0.26 & 0.24 & - \\
    \hline
  \end{tabular}
  \caption{Recall}
\end{table}

\subsubsection{Модели классификации}

\begin{table}[H]
  \centering
  \begin{tabular} { | c | c | }
    \hline
    Модель & Score \\
    \hline
    Bernoulli Naive Bayes & 0.053 \\
    Random Forest & 0.038 \\
    Stochastic Gradient Descent 0.07 & \\
    XGBoosting & 0.02 \\
    \hline
  \end{tabular}
  \caption{Модели классификации}
\end{table}
